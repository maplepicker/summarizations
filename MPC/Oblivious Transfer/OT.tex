\documentclass{article}

\usepackage{amsmath}
\usepackage{indentfirst}
\setlength{\parindent}{2em}

\begin{document}

\section{1-out-of-2 OT}

\par Let the two participants be $P_1$ and $P_2$, $P_1$ has two messages $m_0, m_1$, $P_2$
has a choice bit $b\in \{ 0,1 \}$.

\par Now $P_1$ wants $P_2$ knows one of the messages, but doesn't know the other one.
And $P_2$ wants to receive the message with the choice bit $b$, and $P_1$ knows nothing
about $b$.

\subsection{method based on \textbf{Discrete Logarithm Problem}}

\par \textbf{Basic Setting.} $P_1$ and $P_2$ both know an big integer $g$ and
a big prime number $p$.

\par \textbf{Step1.} $P_1$ generates random keys $s, r_0, r_1 \in_R \{0,1\}^\kappa$,
and $P_2$ generates a random key $k \in_R \{0,1\}^\kappa$.

\par \textbf{Step2.} $P_1$ calculates $g^s$ and send it to $P_2$.

\par \textbf{Step3.} $P_2$ calculates $L=
    \left\{
        \begin{aligned}
            g^k, & b=0 \\
            g^{s-k}, & b=1
        \end{aligned}
    \right.$, and then send it to $P_1$.

\par \textbf{Step4.} $P_1$ calculates $C_0$ and $C_1$, in that
$C_0 = (g^{r_0}, L^{r_0}\oplus m_0)$, $C_1 = (g^{r_1}, ({g^s}/L)^{r_1}\oplus m_1)$
, and then $P_0$ sends $C_0$ and $C_1$ to $P_2$.

\par \textbf{Step5.} $P_2$ reconstructs the result based on received information:

\begin{equation}
    m_b={C_b[0]}^k\oplus C_b[1]
\end{equation}

 
\end{document}